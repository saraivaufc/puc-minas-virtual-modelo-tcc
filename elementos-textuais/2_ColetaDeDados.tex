\chapter{Coleta de Dados}
 
Nessa seção você deve deixar claro onde obteve os dados, o formato e estrutura dos datasets, o relacionamento entre os datasets utilizados, etc. Caso os dados tenham sido obtidos na internet, informe a data e o link em que os dados foram obtidos. Sugere-se que você crie uma tabela com a descrição de cada campo/coluna do seu dataset conforme o exemplo a seguir: 

\begin{table}[h!]
  \begin{center}
    \caption{Descrição dos campos/colunas dos datasets.}
    \label{tab:table1}
    \begin{tabular}{|l|c|r|} % <-- Alignments: 1st column left, 2nd middle and 3rd right, with vertical lines in between
      \hline
       \textbf{Value 1} & \textbf{Value 2} & \textbf{Value 3}\\
      \hline
      1 & 1110.1 & a\\
      \hline
      2 & 10.1 & b\\
      \hline
      3 & 23.113231 & c\\
      \hline
    \end{tabular}
  \end{center}
\end{table}